\documentclass[10pt]{article}
\usepackage{graphicx}  % For including the image
\usepackage{geometry}  % For setting margins
\usepackage{fancyhdr}  % Optional, for setting a persistent header/footer
\usepackage{enumitem} 
\usepackage{pgfgantt} 
\usepackage[table]{xcolor}  
\usepackage{colortbl}      
\usepackage[document]{ragged2e}
\usepackage{hang}

\geometry{a4paper, margin=0.8in}
\setlength\parindent{24pt}

% --- Custom Footer Setup ---
\pagestyle{fancy}
\fancyhf{} % Clear headers/footers
\fancyfoot[L]{\thepage\ / Volume 1, 2025} 
\fancyfoot[R]{FIBO FRA333 – Kinematics Final Project}
\renewcommand{\headrulewidth}{0pt}
\renewcommand{\footrulewidth}{0.5pt}
% ---------------------------

\begin{document}

\noindent
\begin{tabular}{p{0.90\textwidth} c} 
    \parbox[t]{0.90\textwidth}{ 
        \raggedright 
        Institute of Field Robotics, King Mongkut's University of Technology Thonburi \\
        126 Pracha Uthit Rd, Bang Mot, Thung Khru, Bangkok, Thailand 10140 \\
        Senior Thesis FRAB (FIBO Robotics and Automation: Bachelor) \\
        Copyright $\copyright$ 2021 by FIBO
        \vspace{0.5cm}
    }
    & 
    \raisebox{-1.4cm}
    {
        \includegraphics[width=0.7in, keepaspectratio]{img/fibo_logo.png}
    }
\end{tabular}

\noindent
\rule{\textwidth}{1pt}

\vspace{0.4cm}
\noindent
\Large {\textbf{A study of Torque-Control system on UR5 for the drawing task}}

\noindent
\rule{\textwidth}{0.3pt}

\begin{minipage}[t]{0.40\textwidth}
    \raggedright
    \normalsize
    Jirattanun Leeudomwong, 66340500009 \\
    Bharut Aursudkit, 66340500040 \\
    Nuttapruet Puttiwarodom, 66340500018 \\
\end{minipage}
\hfill
\vline
\begin{minipage}[t]{0.60\textwidth}
    \raggedright
    \normalsize
    \begin{quote}
        \textbf{Abstract} \\
        Industrial robot manipulators have countless applications nowadays. One of the basic tasks to test the accuracy of a robot manipulator is to perform a drawing task, which can be done with only position control alone. This project focuses on using a UR5 industrial robot manipulator to perform the drawing task. The challenge this project represents is controlling the force between the pen (end-effector) and the canvas using torque control while following a trajectory generated by the image processing method. This project is running on Gazebo Ignition physics simulation and using ROS2 as a middleware.
        \vspace{0.5cm}

        \textbf{Keywords: Torque control, Dynamics model, Trajectory generation, Image processing, Edge detection, ROS2, Physic simulation}
    \end{quote}
\end{minipage}

\vspace{0.4cm}
\noindent
\rule{\textwidth}{1pt}
\vspace{-1.2cm}
% ------------------- context zone ------------------- 
\section{Overview}

\normalsize
This topic will explain about what do the contribution of this work and what each topic will cover

\section{Trajectory generation}

This topic will explain about the methodology of trajectory generation and result 

\newpage

\section{Inverse Kinematic}

The Inverse Kinematics (IK) used to calculates the joint angles of the UR5e robotic arm to achieve the desired end-effector position and orientation.

This section describes the process for creating both IK and differential kinematics pathways that convert desired Cartesian positions (including velocity and acceleration) into a joint space. The IK implementation can be used in an ROS2 node that publishes a target for each of the robot's joints to that robot's downstream controller.

\subsection{Inverse Kinematic Method}

The UR5e is a 6-DoF serial manipulator with the joint layout:

\begin{enumerate}
    \item Base rotation
    \item Shoulder
    \item Elbow
    \item Wrist 1
    \item Wrist 2
    \item Wrist 3
\end{enumerate}

This structure enables a closed-form inverse kinematics solution, which is implemented in the Python library ur\_analytic\_ik.

The ur\_analytic\_ik solver uses the UR5e’s modified Denavit–Hartenberg (DH) parameters which are:

\begin{table}[h!]
\centering
\begin{tabular}{c c c c}
\hline
\textbf{Joint} & $a_i$ (m) & $d_i$ (m) & $\alpha_i$ (rad) \\
\hline
1 & 0.0     & 0.1625  & $+\pi/2$ \\
2 & -0.425  & 0.0     & $0$      \\
3 & -0.3922 & 0.0     & $0$      \\
4 & 0.0     & 0.1333  & $+\pi/2$ \\
5 & 0.0     & 0.0997  & $-\pi/2$ \\
6 & 0.0     & 0.0996  & $0$      \\
\hline
\end{tabular}
\caption{DH parameters of the UR5e robot.}
\end{table}

The UR analytic IK uses classical geometric IK decomposition which divided into 4 main steps:
\begin{enumerate}
    \item \textbf{Solve Wrist Center Decomposition:} Compute the position of the wrist center by removing the effect of the tool extension along the end-effector z-axis:
    \[
    \mathbf{WC} = \mathbf{P}_{EE} - d_6 \cdot \mathbf{R}_{EE} \hat{z}
    \]
    where $\mathbf{P}_{EE}$ and $\mathbf{R}_{EE}$ are the desired end-effector position and orientation, and $d_6$ is the wrist to tool distance.
    \item \textbf{Solve Joint 1:} Determine the base rotation angle to align the arm with the wrist center:
    \[
    \theta_1 = \arctan2(WC_y, WC_x)
    \]
    \item \textbf{Solve Joints 2 and 3:} Use planar geometry and the law of cosines to compute the shoulder and elbow joint angles. Let $a_2$ and $a_3$ be the link lengths:
    \[
    \theta_2 = \arctan2(z', r') - \arctan2(a_3 \sin\theta_3, a_2 + a_3 \cos\theta_3), \quad
    \theta_3 = \arccos\left(\frac{r^2 - a_2^2 - a_3^2}{2 a_2 a_3}\right)
    \]
    where $(r', z')$ is the projection of the wrist center onto the plane of joints 2 and 3.

    \item \textbf{Solve Wrist Joints (4, 5, 6):} Compute the final three joint angles to achieve the desired end-effector orientation:
    \[
    {}^3R_6 = ({}^0R_3)^\mathsf{T} \cdot {}^0R_6
    \]
    and the wrist angles $\theta_4, \theta_5, \theta_6$ are extracted from ${}^3R_6$ using standard Euler-angle decomposition.
\end{enumerate}

And then each geometry of the manipulator and the properties of trigonometric functions give us 2 option, so the total solution for inverse kinematic is 2(elbow) × 2(shoulder) × 2(wrist) = 8 possible solutions.\vspace{0.5cm}

Then we select the best solution by filtering only solution with elbow up pose then choose the solution with the configuration closest to the previous solution.

\subsection{Inverse Kinematic verification}
To verify inverse kinematic we used forwark kinematic that already coded in this library which is:
\[
{}^{0}\!T_{6}
=
{}^{0}\!T_{1}
\,{}^{1}\!T_{2}
\,{}^{2}\!T_{3}
\,{}^{3}\!T_{4}
\,{}^{4}\!T_{5}
\,{}^{5}\!T_{6}
\]

where each transform is:
\[
{}^{i-1}\!T_i =
\begin{bmatrix}
\cos\theta_i & -\sin\theta_i\cos\alpha_i & \sin\theta_i\sin\alpha_i & a_i\cos\theta_i \\
\sin\theta_i & \cos\theta_i\cos\alpha_i  & -\cos\theta_i\sin\alpha_i & a_i\sin\theta_i \\
0            & \sin\alpha_i              & \cos\alpha_i              & d_i \\
0            & 0                          & 0                         & 1
\end{bmatrix}
\]

and then compare it with the task space we used for inverse kinematic.

\subsection{Differential Inverse Kinematics}
To compute both joint velocities and acceleration, we need UR5e Jacobian matrix. And since ur\_analytic\_ik doesn't have Jacobiancomputation included. Pinocchio library is used for Jacobian computations.

\subsubsection{Joint Velocities}
The joint velocities are calculated from the desired end-effector velocity using the UR5e Jacobian matrix for a given joint configuration.

End-effector velocity $\mathbf{V}$ is related to the joint velocities $\dot{\mathbf{q}}$ as:
\[
\mathbf{V} = J(\mathbf{q}) \, \dot{\mathbf{q}}
\]
Rearranging gives the joint velocities:
\[
\dot{\mathbf{q}} = J^\dagger \mathbf{V}
\]

\subsubsection{Joint Accelerations}
The joint accelerations are calculated from the desired end-effector acceleration using the change in the Jacobian over time:
\[
\mathbf{A} = J(\mathbf{q}) \, \ddot{\mathbf{q}} + \dot{J}(\mathbf{q}, \dot{\mathbf{q}}) \, \dot{\mathbf{q}}
\]
Rearranging gives the joint accelerations:
\[
\ddot{\mathbf{q}} = J^\dagger \left( \mathbf{A} - \dot{J} \, \dot{\mathbf{q}} \right)
\]


\newpage

\section{Dynamics and Control system}

There are various type of control method available for UR5 with torque control. This work presents a motion control strategy based on the dynamics of the UR5. To generate the required torque for motion control, The inverse dynamics equation in joint space (equation 1) is used to compute required torque respect to the Joint state\footnote{Joint state: joint position, joint velocity, joint acceleration}. To control position, velocity and acceleration of the UR5 using PD motion controller (equation 2)

\begin{equation}
    \tau = M(q)\ddot{q} + C(q, \dot{q}) + g(q)
    \label{eq:Dynamics}
\end{equation}

\begin{equation}
    \ddot{q} = \ddot{q_d} + K_p(q_d - q) + K_d(\dot{q}_d - \dot{q})
    \label{eq:pd_control}
\end{equation}

where:
\begin{itemize}
    \item $\tau$ is the joint torque Vector
    \item $M(q)\ddot{q}$ is the Inertia Matrix
    \item $C(q, \dot{q})$ is the Coriolis and Centrifugal Matrix
    \item $g(q)$ is the Gravity Vector
    \item $\ddot{q}$ is the computed acceleration (command)
    \item $K_p$ and $K_d$ are the proportional and derivative gain matrices
    \item $q_d$ and $q$ are the desired and actual joint positions
    \item $\dot{q}_d$ and $\dot{q}$ are the desired and actual joint velocities
    \item $\ddot{q_d}$ is the desired joint acceleration
\end{itemize}

However, In the dynamics model in joint space (equation 1) and PD controller (equation 2) alone are not able to control the force at the end-effector. The method for control motion and force at end-effector simultaneously is to apply the Hybrid force-motion control. Even so the hybrid force-motion control is using the dynamics model in task space. Make the dynamics calculation more complicated to compute. Even if using dynamics model for joint space for motion controller and force controller together. Controller will be fighting over position. This project presents the Z-axis elevation PI control (equation 3) method to control the force occurs at the end-effector. F/T sensor is noisy sensor even apply the low-pass filter. This cause the force elevation controller is only PI controller because the D term is noist sensor. This method will not affect the control system to compensate the position error with stronger force respect to the time because of the motion controller is only PD controller. 

\begin{equation}
    Z = -1 \cdot (K_p(W_{d} - W) + K_i \int(W_{d} - W) \frac{d}{dt})
    \label{eq:z-control}
\end{equation}

where:
\begin{itemize}
    \item $Z$ is the elevation in Z axis (task space)
    \item $K_p$ and $K_i$ are the proportional and integral gain matrices
    \item $W_d$ and $W$ are the desired and actual force at end-effector Z axis
\end{itemize}

After apply PD motion controller and PI elevation controller, UR5 are able to control motion and force simultaneously without fighting between 2 controllers in joint space. The control diagram is show in Fig.\ref{fig:control_system_diagram}

\begin{figure}[h]
    \centering
    \includegraphics[width=7cm]{img/control_system_diagram.png}
    \caption{Control system diagram}
    \label{fig:control_system_diagram}
\end{figure}

\newpage

Lastly the inverse dynamics model calculation done by pinocchio framework on python. And Controlled by the custom code by this project team. (code implimentation here Github: )


\section{Test and Evaluation}

The requirement of this project is to perform drawing task within position error 10\% range. To evaluate the

\input{sub_tex/conclusion.tex}


\end{document}